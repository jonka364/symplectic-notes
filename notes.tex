% !TEX TS-program = pdflatex
% !TEX encoding = UTF-8 Unicode

% This is a simple template for a LaTeX document using the "article" class.
% See "book", "report", "letter" for other types of document.

\documentclass[11pt]{article} % use larger type; default would be 10pt

\usepackage[utf8]{inputenc} % set input encoding (not needed with XeLaTeX)

%%% Examples of Article customizations
% These packages are optional, depending whether you want the features they provide.
% See the LaTeX Companion or other references for full information.

%%% PAGE DIMENSIONS
\usepackage{geometry} % to change the page dimensions
\geometry{a4paper} % or letterpaper (US) or a5paper or....
% \geometry{margin=2in} % for example, change the margins to 2 inches all round
% \geometry{landscape} % set up the page for landscape
%   read geometry.pdf for detailed page layout information

\usepackage{graphicx} % support the \includegraphics command and options

% \usepackage[parfill]{parskip} % Activate to begin paragraphs with an empty line rather than an indent

%%% PACKAGES
\usepackage{booktabs} % for much better looking tables
\usepackage{array} % for better arrays (eg matrices) in maths
\usepackage{paralist} % very flexible & customisable lists (eg. enumerate/itemize, etc.)
\usepackage{verbatim} % adds environment for commenting out blocks of text & for better verbatim
\usepackage{subfig} % make it possible to include more than one captioned figure/table in a single float
% These packages are all incorporated in the memoir class to one degree or another...

\usepackage{amsmath}
\usepackage{amssymb}
\usepackage{mathrsfs}

%%% HEADERS & FOOTERS
\usepackage{fancyhdr} % This should be set AFTER setting up the page geometry
\pagestyle{fancy} % options: empty , plain , fancy
\renewcommand{\headrulewidth}{0pt} % customise the layout...
\lhead{}\chead{}\rhead{}
\lfoot{}\cfoot{\thepage}\rfoot{}

%%% SECTION TITLE APPEARANCE
\usepackage{sectsty}
\allsectionsfont{\sffamily\mdseries\upshape} % (See the fntguide.pdf for font help)
% (This matches ConTeXt defaults)

%%% ToC (table of contents) APPEARANCE
\usepackage[nottoc,notlof,notlot]{tocbibind} % Put the bibliography in the ToC
\usepackage[titles,subfigure]{tocloft} % Alter the style of the Table of Contents
\renewcommand{\cftsecfont}{\rmfamily\mdseries\upshape}
\renewcommand{\cftsecpagefont}{\rmfamily\mdseries\upshape} % No bold!

%%% END Article customizations

%%% The "real" document content comes below...

\title{Symplectic geometry and classical mechanics}
\author{Jonas Karlsson}
%\date{} % Activate to display a given date or no date (if empty),
         % otherwise the current date is printed 

\begin{document}
\maketitle
These are my notes on symplectic geometry and classical mechanics, based on a course by Tobias Osborne given in the winter semester 2017-2018 at the Leibniz Universität Hannover. I watched the lectures on YouTube, and I strongly recommend them. Not only might these notes contain misunderstandings introduced by me (and for which I take responsibility); they are also not particularly complete, since I made no attempt to write down everything that was said in the lectures. Rather I made notes for my own use, selectively writing down what I felt I needed to read later. Caveat lector.

\section*{Notation}

\section*{}

\section*{}

\section*{}

\section*{Lecture 6}

\section*{}

\section*{Symplectic forms and manifolds}
Let $V$ be a real vector space and let $\Omega: V\times V \rightarrow \mathbb{R}$ be a bilinear skew-symmetric form. A Gram-Schmidt argument shows that $V$ admits a basis $\{u_1, \dots, u_k, e_1, \dots, e_n, f_1, \dots, f_n\}$, for which
\begin{align*}
 \Omega(u_i, \cdot)  &= 0  \\
 \Omega(e_i, e_j) &= \Omega(f_i, f_j) = 0 \\
\Omega(e_i, f_j) &= \delta_{ij}
\end{align*}
for all $i$ and $j$; in other words, the $\{u_i\}$ span the degenerate part and the $\{e_i\}$ and $\{f_i\}$ are a standard basis for the non-degenerate part. They should remind you of position and momentum in quantum mechanics. We have that $k + 2n = \operatorname{dim}V$. If $k=0$, $\Omega$ is non-degenerate and is called symplectic. Note that this can only happen if $ \operatorname{dim}V$ is even. Recall what non-degeneracy means for forms on a vector space: given any form (linear map) $\beta: V\times V \rightarrow \mathbb{R}$ we can curry and get a map
$$
\tilde{\beta}_1: V \rightarrow V^\ast,
$$
the exponential transpose, given by
$$
\tilde{\beta}_1(u)(v) = \beta(u,v)
$$
(and similarly $\tilde{\beta}_2$ by currying in the second slot first). Thus there are exact sequences
$$
0 \rightarrow \operatorname{ker}\tilde{\beta}_i \rightarrow V \rightarrow V^\ast \rightarrow \operatorname{coker}\tilde{\beta}_i \rightarrow 0.
$$
If $V$ is finite-dimensional, then if either of the $\tilde{\beta}_i$ is injective, then so is the other, and $\beta$ is called nondegenerate. Then each $\tilde{\beta}_i$ is an injective map from a finite-dimensional vector space to another vector space of the same dimension, hence an isomorphism (of $V$ with its dual $V^\ast$). None of this uses any symmetry assumption on $\beta$. Returning to the case of $\Omega$, we notice that the skew-symmetry means that $\tilde{\beta}_1 = - \tilde{\beta}_2$, so that it does little difference in which slot we curry. What is important is that a non-degenerate form gives a way to identify a vector space with its dual; this will be used extensively in what follows. Turning to differential forms, a $2$-form $\omega$ on a manifold $M$ is called symplectic if it is closed ($d\omega = 0$) and each $\omega_p$ is a symplectic form on the tangent space $T_pM$, for every point $p\in M$. The data $(M, \omega)$ of a manifold and a symplectic $2$-form is called a symplectic manifold. A choice of $\omega$ on a given $M$ is called a symplectic structure. Since non-degenerate symplectic forms only exist on even-dimensional vector spaces, only even-dimensional manifolds admit a symplectic structure. This condition is necessary but not sufficient. For example, the standard area $2$-form on the $2$-sphere $S^2$ is symplectic; it is non-degenerate, and is automatically closed since $S^2$ is two-dimensional. In fact, no other sphere admits a symplectic structure.\\
Finally, we may ask whether the standard vector space basis exhibited initially in this section extends to open sets on manifolds. It does, and this will be dealt with in lecture 17. For now we merely state that the corresponding coordinates exist and are called Darboux coordinates.


\section*{Lecture 9: Cotangent bundles are symplectic}
Let $M$ be a smooth manifold. Then its cotangent bundle $T^\bullet M$ is again a smooth manifold. If the functions $x_i$ are coordinates on (some subset of) $M$, then a section of $T^\bullet M$ can be written uniquely as
$$
\xi = \sum_i \xi_i(x_1, \dots, x_n) dx_i,
$$
where the $\xi_i$ are smooth functions of the $x_i$. Here we are thinking of the cotangent bundle as a sheaf of $1$-forms. To get the total space of the bundle, we promote the $\xi_i$ to coordinates. That is, we introduce a new space of twice the dimension, whose points are pairs $(p, \xi_p)$ of a point $p\in M$ and a form $\xi_p \in T^\bullet_p M$ and on which the $\xi_i$ are functions. That is, coordinates on $T^\bullet M$ are tuples $(x_1, \dots, x_n, \xi_1, \dots, \xi_n)$, and a section of the projection to $M$ is a $1$-form in accordance with the equation above. 

It then remains to work out how they behave under changes of coordinates. By definition, two $1$-forms are equal if they are equal as maps from tangent vectors to numbers. 


Next, we claim that there is an intrinsically defined $1$-form
$$
\alpha = \sum_i \xi_i dx_i,
$$
called the tautological $1$-form. As an example, let's take $M=\mathbb{R}^2$ away from the origin and let's work out $\alpha$ in both Cartesian and polar coordinates. In Cartesian coordinates we have
$$
\alpha = \xi_x dx + \xi_y dy
$$

Given that $\alpha$ is intrinsically defined, we obtain an equally intrinsic symplectic form $\omega = -d\alpha$, given in coordinates by
$$
\omega = \sum_i dx_i \wedge d\xi_i.
$$
That $\omega$ is closed is immediate, since it is exact by definition. It is also non-degenerate (this can by checked in any chart).
\section*{Lecture 10: Vector fields and flows}

\section*{Lecture 11: Properties of the Lie derivative}

\section*{Lecture 12: The interior product}


\section*{Lecture 13: Symplectic fields and flows}
Let $X$ be a vector field on $(M,\omega)$. Then $X$ is called symplectic if $\mathscr{L}_X \omega = 0$. Recall that a vector field is called Hamiltonian if it arose from a function $H$ on $M$ via
$$
\omega (X_H, \cdot) = dH.
$$
By a calculation, any Hamiltonian field is symplectic. More specifically, let $i_X\omega = \alpha_X$. Then
$$
d\alpha_X = \mathscr{L}_X \omega.
$$
Hence a vector field $X$ is symplectic if and only if its corresponding $1$-form $\alpha_X$ is closed, and it is Hamiltonian if and only if $\alpha_X$ is exact. Thus the relation between symplectic and Hamiltonian vector fields is precisely the relation between closed and exact $1$-forms. We know from de Rham theory that this hinges on the topology of $M$, specifically $H^1(M)$. As for flows, $\operatorname{exp}(tX)$ preserves $\omega$ if and only if $X$ is symplectic. \\
Turning now to the special case of cotangent bundles, suppose $M$ is an arbitrary manifold and $X$ a vector field on $M$. Then there is a unique lift of $X$ to a vector field $X_\#$ on $T^\bullet M$, and this field is always Hamiltonian with 
$$
H = i_{X_\#}\alpha,
$$
where $\alpha$ is the tautological $1$-form on $T^\bullet M$.


\section*{Lecture 14: Lagrangian submanifolds}


\section*{Lecture 15: The adjoint and coadjoint representations}

\section*{Lecture 16: The moment map} 

\section*{Lecture 17: The Darboux-Weinstein theorem}
This states that symplectic manifolds have no local invariants beyond the dimension. 

\section*{Lecture 18: The Marsden-Weinstein-Meyer theorem}
If a sympectic manifold is equipped with a continuous group action, we can sometimes take a quotient by the group action and obtain a smaller symplectic manifold. Suppose the symplectic manifold $(M,\omega)$ is acted on symplectically by a Lie group $G$, that is, there is a smooth group homomorphism $\psi: G \rightarrow \operatorname{Sympl}(M)$. Suppose also that there is a compatible moment map $\mu: M \rightarrow \mathfrak{g}^\ast$ to the dual of the Lie algebra of $G$, compatibility meaning that
$$
\mu \circ \psi (g) = \operatorname{Ad}^\ast_g \circ \mu
$$
for every group element $g$.
(TODO: find that comment about the cohomology assumptions on G)
Then $G$ stabilizes the level set $\mu^{-1}(0) \subset M$


\section*{Lecture 19: Properties of the moment map}
\subsection*{The Atiyah-Guillemin-Sternberg convexity theorem}
Let $(M, \omega)$ be a compact, connected symplectic manifold with a Hamiltonian action by a torus $G = T^m$ and let $\mu: M \rightarrow \mathfrak{g}^\ast$ be the moment map. Then:
\begin{enumerate}
\item The level sets of $\mu$ are connected;
\item The image of $\mu$ is the convex hull of the images of the fixed points of the action.
\end{enumerate}
In view of this, the image of $\mu$ is called the \emph{moment polytope} of the action. Note that $\mathfrak{g}^\ast \cong \mathbb{R}^n$ is a real vector space so the notion of convex hull makes sense. 

\subsection*{Liouville measure}
On a symplectic manifold $(M, \omega)$, the top exterior power $\omega^n = \wedge^n \omega$ is nondegenerate, hence a volume form. The normalized quantity $\omega^n/n!$ is called the Liouville measure. If $X$ is a Hamiltonian vector field on $M$, so that $\mathscr{L}_X \omega = 0$. Then (by Leibniz) $\mathscr{L}_X \omega^n = 0$ also, so Hamiltonian flows preserve the Liouville measure. In particular this is true for the time evolution of a dynamical system, which is Liouville's theorem. 
\subsection*{Duistermaat-Heckman measure}
Suppose that $M$ is compact so that in particular 
$$
\int_M \frac{\omega^n}{n!} 
$$
is finite. Then for any moment map $\mu:M \rightarrow \mathbb{R}$ we can push Liouville forward to a measure on $\mathbb{R}$, called the Duistermaat-Heckman measure. 

Now suppose that we have a circle action on $M$, i.e. that the flow $\sigma_t = \operatorname{exp}(tX)$ is periodic. 


\section*{Lecture 20: Almost complex and complex manifolds}

\section*{Lecture 21: Kähler manifolds and the geometry of quantum mechanics}
\end{document}

\section*{}