% !TEX TS-program = pdflatex
% !TEX encoding = UTF-8 Unicode

% This is a simple template for a LaTeX document using the "article" class.
% See "book", "report", "letter" for other types of document.

\documentclass[11pt]{article} % use larger type; default would be 10pt

\usepackage[utf8]{inputenc} % set input encoding (not needed with XeLaTeX)

%%% Examples of Article customizations
% These packages are optional, depending whether you want the features they provide.
% See the LaTeX Companion or other references for full information.

%%% PAGE DIMENSIONS
\usepackage{geometry} % to change the page dimensions
\geometry{a4paper} % or letterpaper (US) or a5paper or....
% \geometry{margin=2in} % for example, change the margins to 2 inches all round
% \geometry{landscape} % set up the page for landscape
%   read geometry.pdf for detailed page layout information

\usepackage{graphicx} % support the \includegraphics command and options

% \usepackage[parfill]{parskip} % Activate to begin paragraphs with an empty line rather than an indent

%%% PACKAGES
\usepackage{booktabs} % for much better looking tables
\usepackage{array} % for better arrays (eg matrices) in maths
\usepackage{paralist} % very flexible & customisable lists (eg. enumerate/itemize, etc.)
\usepackage{verbatim} % adds environment for commenting out blocks of text & for better verbatim
\usepackage{subfig} % make it possible to include more than one captioned figure/table in a single float
% These packages are all incorporated in the memoir class to one degree or another...

\usepackage{amsmath}
\usepackage{amssymb}
\usepackage{mathrsfs}

%%% HEADERS & FOOTERS
\usepackage{fancyhdr} % This should be set AFTER setting up the page geometry
\pagestyle{fancy} % options: empty , plain , fancy
\renewcommand{\headrulewidth}{0pt} % customise the layout...
\lhead{}\chead{}\rhead{}
\lfoot{}\cfoot{\thepage}\rfoot{}

%%% SECTION TITLE APPEARANCE
\usepackage{sectsty}
\allsectionsfont{\sffamily\mdseries\upshape} % (See the fntguide.pdf for font help)
% (This matches ConTeXt defaults)

%%% ToC (table of contents) APPEARANCE
\usepackage[nottoc,notlof,notlot]{tocbibind} % Put the bibliography in the ToC
\usepackage[titles,subfigure]{tocloft} % Alter the style of the Table of Contents
\renewcommand{\cftsecfont}{\rmfamily\mdseries\upshape}
\renewcommand{\cftsecpagefont}{\rmfamily\mdseries\upshape} % No bold!

%%% END Article customizations

%%% The "real" document content comes below...

\title{Symplectic geometry and classical mechanics}
\author{Jonas Karlsson}
%\date{} % Activate to display a given date or no date (if empty),
         % otherwise the current date is printed 

\begin{document}
\maketitle
These are my notes on symplectic geometry and classical mechanics, based on a course by Tobias Osborne given in the winter semester 2017-2018 at the Leibniz Universität Hannover. I watched the lectures on YouTube, and I strongly recommend them. Not only might these notes contain misunderstandings introduced by me (and for which I take responsibility); they are also not particularly complete, since I made no attempt to write down everything that was said in the lectures. Rather I made notes for my own use, selectively writing down what I felt I needed to read later. Caveat lector.


\section*{Lecture 16: The moment map} 

\section*{Lecture 17: The Darboux-Weinstein theorem}
This states that symplectic manifolds have no local invariants beyond the dimension. 

\section*{Lecture 18: The Marsden-Weinstein-Meyer theorem}

\section*{Lecture 19: Properties of the moment map}
\subsection*{The Atiyah-Guillemin-Sternberg convexity theorem}
Let $(M, \omega)$ be a compact, connected symplectic manifold with a Hamiltonian action by a torus $G = T^m$ and let $\mu: M \rightarrow \mathfrak{g}^\ast$ be the moment map. Then:
\begin{enumerate}
\item The level sets of $\mu$ are connected;
\item The image of $\mu$ is the convex hull of the images of the fixed points of the action.
\end{enumerate}
In view of this, the image of $\mu$ is called the \emph{moment polytope} of the action. Note that $\mathfrak{g}^\ast \cong \mathbb{R}^n$ is a real vector space so the notion of convex hull makes sense. 

\subsection*{Liouville measure}
On a symplectic manifold $(M, \omega)$, the top exterior power $\omega^n = \wedge^n \omega$ is nondegenerate, hence a volume form. The normalized quantity $\omega^n/n!$ is called the Liouville measure. If $X$ is a Hamiltonian vector field on $M$, so that $\mathscr{L}_X \omega = 0$. Then (by Leibniz) $\mathscr{L}_X \omega^n = 0$ also, so Hamiltonian flows preserve the Liouville measure. In particular this is true for the time evolution of a dynamical system, which is Liouville's theorem. 
\subsection*{Duistermaat-Heckman measure}
Suppose that $M$ is compact so that in particular 
$$
\int_M \frac{\omega^n}{n!} 
$$
is finite. Then for any moment map $\mu:M \rightarrow \mathbb{R}$ we can push Liouville forward to a measure on $\mathbb{R}$, called the Duistermaat-Heckman measure. 

Now suppose that we have a circle action on $M$, i.e. that the flow $\sigma_t = \operatorname{exp}(tX)$ is periodic. 


\section*{Lecture 20: Complex manifolds}

\end{document}
